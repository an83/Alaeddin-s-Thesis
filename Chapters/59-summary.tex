\pagebreak
\section{Social Data Summary}
% Tobias: - The whole interpretation of the results and discussion for the next user study and the overall chapter is very thin and is urgently needed. What do we know after this chapter? What knowledge/insights have we gained with respect to the overall aim of this thesis. What are the gaps? What are the limitations? What is the gap which will be closed in the next chapter (bridge to the next chapter)?

This chapter explored different options for representing social data (e.g., 360-degree images, 3D scanned room) in a wearable social AR interface. The explored options include: 1) filtering the type of shared social data (Section \ref{sec:surrounding:360}), 2) filtering the level of detail of 3D shared surrounding environments (Section \ref{sec:surrounding:environment}), and 3) filtering partial elements in shared social data (Section \ref{sec:surrounding:hiding}). 

The user studies in this chapter showed that 1) when sharing 360-degree videos, viewers preferred (in ranking conditions) to use physical walking toward/away from the social contacts as a way to change the level of detail of the shared social data over no filter, 2) when sharing a 3D surrounding environments, the perceived comfort in terms of privacy for both viewer and sharer was higher when having a social proximity filer over no filter, and 3)  When sharing a 3D surrounding environment, it is more comfortable and secure when a social filter is applied over no filter.

These results of this chapter validate the different levels in the dimension of sharing social data on the Social AR Continuum. The next chapter looks into the interactions between social contacts and shared social data. 