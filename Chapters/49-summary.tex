\pagebreak
\section{Social Contacts Summary}

This chapter explored different options of visualising and representing social contacts on wearable AR devices. Options include: 1) changing the visual fidelity of avatars based on the social relationship, and 2) changing the scale and placement of social contacts based on the social relationship. Two user studies were conducted on both dimensions to validate their effect on social presence and usability. 

The user studies show that viewers preferred visual fidelity and social proximity over no social filter. They also showed no difference between displaying social contacts as life-sized or miniature placement. This validates the hypothesis that representations of social contacts can be created based on social proximity between social contacts. 

Initial implementation in this chapter focused on the representation of the social contact "friend" rather than on the interaction with the avatar. In Section \ref{sec:surrounding:hiding}, that focuses on the interaction between a Viewer and a Sharer, the avatar body position and rotation are updated based on the real person relative position in the shared room. Facial expressions can be added as well as a separate dimension that would include different levels of detail of interactivity of the virtual avatar (e.g., static 3D avatar, updating body pose 3D avatar, updating facial expression 3D avatar).

Future explorations in visualising social contacts may include placing social contacts on the user's palm or make avatars follow the user if the user moves in space. 

The following chapter \ref{ch:data} will move to explore different dimension on the Social AR Continuum, focusing on the shared social data and the shared surrounding environments between social contacts, and how these are represented in AR. 

% your summary is not summarising the results or finding. The most important thing for you to do with this chapter is to have a proper discussion section where you discuss the results. What do they mean, are there any confounding factors that could have caused these results. Can they be generalised (external validity)? What are other limitations/shortcomings of this study? What is the overall take away message? What is the relevance of the next chapters?

% Tobias; you do not report if you used one-tailed or two-tailed tests. Also if I recall correctly, you often do not report on tests for normal distribution, but I hope you did them or do you just assume it is not normally distributed? It is actually highly recommended to check for normal distribution. Because if the results follow a normal distribution, then parametric tests should be used, they yield higher accuracy over using non-parametric test per default. I, as a reviewer, would pick up on this.
