\begin{abstract}
\addchaptertocentry{\abstractname} % Add the abstract to the table of contents

% Abstract: In general, your abstract should give a more complete summary of the thesis (not the work). This means that you should have at least one sentence for each chapter, and there should be more details about the summary of your results. You get close in the last few sentences here, but more detail will be needed. I'm actually confuse that you are writing the Abstract FIRST! Usually it is the LAST thing you write.

The primary goal of this thesis is to develop and evaluate novel interaction techniques for enhancing Augmented Reality (AR) wearable headsets focusing on how they can be used to share social experiences with our family and friends. The motivation for this research comes from the need for more intuitive and natural approaches to sharing our social experiences in life using wearable AR headset that increase our social and co-presences. 
% Rob: Is this software, or the Social Continuum? What kind of Framework do you mean?
With a framework for sharing social experiences on wearable devices including panorama, live-video and 3D surrounding environment
% Rob: Are these three orthogonal? Why list them here? Is this all of them?
, This work discusses advantages and limitations of various implementations and techniques of sharing social experiences using these media.  

Based on Human-Computer Interaction methodologies, this work designed and conducted user studies to test and evaluate user presence and system usability of our implementation in sharing and annotating social experiences on AR headsets. This work developed the concept of the Social AR Continuum which describes the space of sharing experiences on AR in various dimensions. This thesis focused on the essential dimensions of "representing contacts", "sharing data" and "annotation", and developed user interfaces on these dimensions and evaluated them using user studies. 

The user study evaluation results show that using AR to share social experiences can increase users' presence. This work studied various ways to represent social contacts on AR in a focus group and user study, and show the user preference of the interface. This thesis summarises the results as design recommendations to help interface designers for designing shared experience on wearable AR systems. 

% Content: Can you insert in this where each of the user studies will be discussed? Also, don't you think the Social AR Continuum deserves its own chapter? Seems like one of the main topics of the thesis, and something you can attach your name to, so it should be highlighted. Also, there you can outline many possible axes, then go deep on the three you have chosen to explore. This indicates fertile areas that others can explore within the Continuum.

\end{abstract}
