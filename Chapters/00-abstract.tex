\begin{abstract}
\addchaptertocentry{\abstractname} % Add the abstract to the table of contents

The primary goal of this thesis is to explore, develop and evaluate novel interaction techniques for Augmented Reality (AR) wearable headsets, focusing on how they can be used to share social experiences with our family and friends. There is a need in Afor a more intuitive and natural approach to sharing our social experiences in life through wearable AR headsets that increase our social and co-presences. With a design framework for visualising and interacting with social networks on wearable AR devices, This work discusses the advantages and limitations of various implementations and techniques of shared social experiences on wearable AR.  

Based on Human-Computer Interaction methodologies, this work developed and ran user studies to test and evaluate user presence and system usability of our implementation in visualising, sharing and interaction of social experiences on AR headsets. This work developed the concept of the Social AR Continuum, which describes the space of sharing experiences on AR in various dimensions. This thesis focused on the essential dimensions of "representing contacts", "sharing data" and "interactions", and developed user interfaces on these dimensions and evaluated them using user studies. 

The user study evaluation results show that using AR to share social experiences can increase users' social presence. This work studied various ways to represent social contacts on AR in a focus group and user study, and show the user preference of the interface. This thesis summarises the results as design recommendations to help interface designers for designing shared social experiences on wearable AR systems. 

% Content: Can you insert in this where each of the user studies will be discussed? Also, don't you think the Social AR Continuum deserves its own chapter? Seems like one of the main topics of the thesis, and something you can attach your name to, so it should be highlighted. Also, there you can outline many possible axes, then go deep on the three you have chosen to explore. This indicates fertile areas that others can explore within the Continuum.

\end{abstract}
