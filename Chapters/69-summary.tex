\section{Social Interactions Summary}

This chapter explored different options of representing social interactions in AR including samples of 1) sharing annotations on panoramic video, 2) sharing 3D annotation using depth cameras, and 3) sharing 360 panoramic image with annotation and awareness cues. These explorations include user studies measure social presence, usability and user' feedback/preference. These interactions can be used on the Social AR Continuum to control the interactions between social contacts and shared data based on social proximity. For instance, for closer relationship, higher fidelity of social interactions (e.g., 3D annotations and drawing) can be enabled, while lower fidelity (e.g., text list annotations and pointing) is for further away social proximity relationship.

The user study of annotation on live stream video showed that there is a statistical difference in social presence (in particular perceived message understanding and perceived effective understanding) and usability score when higher level of details of interaction and annotation is available for participants. The user study of annotation on 3D depth data showed that there are statistical difference in social presence when using 3D annotation. The last user study showed there are statistical difference on social presence when using higher level of interaction on a 360 panorama using wearable AR display. 

The following chapter will summarise the conclusions of the entire thesis and highlight few future directions from this research.
