\chapter{Introduction} % Main chapter title
\label{ch:intro} % Change X to a consecutive number; for referencing this chapter elsewhere, use \ref{ChapterX}

% In this chapter, we introduce mixed-reality continuum. AR/VR becoming popular. Previous research focus on single users or collaboration with few users. The question remain how AR will be used with many uses and in social situations. The Social AR Continuum can be an answer to this question.

\section{Problem Statement}

% - The potential for AR to be a social sharing platform. \\
% - Sharing social experiences was not fully addressed in the research community

With major industry players (e.g. Microsoft, Facebook and Apple) supporting Augmented Reality (AR), 
it is likely that one day people will soon be using head-worn AR displays on a daily basis. 
One potential use for this technology could be for connecting with social networks. 
Just as people today use their mobile phones to connect with hundreds or thousands of "friends", wearable AR displays could be used to connect with friends and view and interact with their shared information.

Previously, handheld and wearable AR systems have been used to view social networks in a number of different ways. For example, Presslite's Twitter 360\footnote{https://www.youtube.com/watch?v=5w7EAz8-uwU} shows virtual tweets overlaid on the real world at the locations of the people that sent them, and early versions of Junaio\footnote{https://en.wikipedia.org/wiki/Junaio} allowed people to drop virtual messages and pictures in the real world, as did the popular application Sekai Camera\footnote{https://www.youtube.com/watch?v=oxnKOQkWwF8\&t=61s}. Most of these applications were focused on asynchronous collaboration, enabling people to post virtual messages in space which can later be browsed and retrieved by other users. However similar technology could also be used for live synchronous collaboration such as a live video streaming the remote user point of view \cite{Nassani2016}, live video avatar sharing  \cite{Billinghurst2002}, or sharing realistic 3D models superimposed over the real world \cite{Fanello2016}, or by using virtual avatars to show a live view of remote collaborators and their surrounding space as in the Holoportation system  \cite{Fanello2016} which shows a realistic 3D AR representation of a remote collaborator.

% Tobias: Make clear here that scrolling etc is no option as we are constrained by the interactions and would require unobtrusive interactions.
Unlike for handheld devices, by using a wearable AR display like the Microsoft HoloLens\footnote{https://www.microsoft.com/en-nz/hololens}, it could be possible to see an AR representation of the user's social network visible at all times. However, this raise the question of how to visually represent the contacts in the network. For example, if a user has a large social network with hundreds of contacts available, visually representing each of themmight clutter the user's visual space.

In our research we are interested in how to represent a social network with hundreds or thousands of contacts in a wearable AR interface. If there are dozen of virtual tags in an AR view representing people, how can they be distinguished between each other? How will this scale to hundreds or thousands of people? This is an important area of research as it will allow users to view and interact with a large number of social network followers at different levels of privacy and social engagement.

One of the challenges we are seeking to address is how to manage and represent communication from dozens, hundreds or thousands of contacts in a social network in an AR interface. If there are dozens of virtual tags in view representing people on the user's social network, how can they be distinguished between each other? How will this scale to hundreds or thousands of people?

\section{Research Questions}

This thesis targets the future where AR devices are used everyday, and looking into the future of visualising social networks on AR view by addressing the following research questions: 

\begin{itemize}
    \item RQ1: Could we filter information representing social contacts based on social proximity?
    \item RQ2: How should social data be shared on the social proximity continuum?
    \item RQ3: How would people feel in terms of privacy when sharing their surrounding environments with social contacts?
\end{itemize}


\section{Contributions}

\begin{itemize}
    \item Defining the dimensions in design space of sharing social experiences on wearable AR
    \item Technical implementing different prototypes of sharing social experiences on wearable devices
    \item User study evaluations of behaviour and preception of using social AR prototypes.
\end{itemize}


\section{Selected Publications}

The following lists of selected publications were peer-reviewed as part of submission to scientific conferences in AR and HCI. 

\begin{itemize}
    \item{ \fullcite{Nassani2017a}}
    \item{ \fullcite{Nassani2016}}
    \item{ \fullcite{Nassani2015}}
    \item{ \fullcite{Nassani2015a}}    
    \item{ \fullcite{Reichherzer2014}}
    \item{ \fullcite{Billinghurst2014}}
    \item{ \fullcite{Nassani2018a}}
    \item{ \fullcite{Nassani2018b}}
    \item{ \fullcite{Nassani2018c}}
    \item{ \fullcite{Nassani2017b}}
    \item{ \fullcite{Nassani2017}}
\end{itemize}