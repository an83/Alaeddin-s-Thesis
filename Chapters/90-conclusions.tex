\chapter{Conclusions}
\label{ch:conclusions}

We introduced in this thesis the concept of AR Social Continuum in Chapter \ref{ch:continuum} to answer few research questions about representing and interacting with social networks through AR wearable devices. We explored the main dimensions in representing self \& others in Chapter \ref{ch:contacts}, shared data in Chapter \ref{ch:data} and annotation \& interactions in Chapter \ref{ch:annotation}. 

The research focused on exploring how social proximity can be used to filter (i.e., show more/less details of ) the representation of social contacts, shared virtual objects and the surrounding environments through wearable AR devices. In particular, we addressed four research questions about 1) the dimension and variables of the social AR continuum, 2) the representation of self and others as virtual avatars on the social continuum, 3) the representation of data and shared environments, and 4) the interactions and annotations techniques between social contacts.   

To answer the research questions above, we built software prototypes to explore and validate different dimensions on the social AR continuum. We run user studies to evaluate the user interactions with these prototypes, we statistically analysed the results and reported on the quantitative and qualitative results. 

In this chapter we discuss the results of the user studies as well as summarise the general directions for future uses and developments of the AR Social Continuum. 

\pagebreak

\section{Lesson Learned}

For answering \textit{RQ1: How to represent social networks and shared social data on wearable AR devices}, we established the concept and dimensions of the social AR continuum (Chapter \ref{ch:continuum}).

\begin{itemize}
    \item{The dimensions of the social AR continuum can be used to vary the shared social experience based on social proximity. These dimensions include: 1) self and others, 2) shared data and surrounding environments, and 3) interaction and annotations.}
    \item{Future scenarios has been described where using the social AR continuum can help enhancing the shared social AR experience. Those scenarios include: 1) collaborative home decoration, 2) sharing home office, 3) face-2-face social drinks, and 4) social connections at a conference.}
\end{itemize}

For answering \textit{RQ2: How to represent annotations/tags on wearable AR devices for shared social experiences}, we explored three concepts in sharing annotation and interaction for sharing social experiences. We run a user study on sharing social videos (section \ref{sec:video}), and compared three conditions: List, AR, List+AR. We looked into creating and finding 3D annotations for social AR tagging (section \ref{sec:3D}). We looked into sharing 360 panorama for sharing social experiences (section \ref{sec:pano}). 

\begin{itemize}
    \item{There was no statistical significance in usability between List, AR and List+AR conditions for visualising comments on shared social videos. This indicate that all three conditions are equal good usability for the user.}
    \item{There was statistical difference in social presence of perceived message understanding and affective understanding between List and AR. This indicates that there is higher perceived messages and affective understanding in AR and List+AR comparing to List condition}
    \item{There was also a statistical difference in terms of ranking between List and AR. This indicates that users preferred AR conditions over non-AR condition}
    \item{We developed a prototype extending a normal 2D wearable (Google Glass) interface with a 3D sensor (Google Tango) to enable placing social annotation in 3D space}
    \item{We found statistical significance indicating that using our 3D tagging system was easy \& natural to use and was not found to be physically or mentally challenging. Also finding a previously created AR tag was statistically significant in being useful.}
    \item{We developed a prototype connecting a wearable headset with a hand-held interface through networking protocol to share a 360 panoramic image for sharing social experiences.}
    \item{We compared three interaction (audio, pointing and drawing) techniques}
    \item{We found statistical significance results in using Glass for drawing comparing to pointing and audio}
\end{itemize}

For answering \textit{RQ3: How to share virtual objects/data with our social network on wearable AR devices}, we explored filtering shared social 360 videos (section \ref{sec:surrounding:360}), sharing 3D captured surrounding room (section \ref{sec:surrounding:environment}), and the privacy concerns of hiding and showing part of the shared room (section \ref{sec:surrounding:hiding}). 

\begin{itemize}
    \item{We developed several prototypes on HoloLens for sharing and filtering social data such as 360 panorama video and 3D scanned surrounding room. The filtering is based on social proximity between social contacts.}
    \item{We found statistical significance confirming that filtering the shared social data is more preferred in terms of social presence.}
    \item{We found that perceived comfort in terms of privacy was statistically significant when we proximity filter was applied comparing to no filter.}
    \item{We found statistical significance in co-presence and comfort between proximity filter and no filter}
    \item{We found that for sharer that the privace of shared spaces is more important than for the viewer.}
    \item{We found that the hiding mechanism had no significant difference between 1) remove, 2) blur and 3) overlay. However the most preferred mechanism of hiding shared social data is to remove them completely from the viewer view.}
\end{itemize}

For answering \textit{RQ4: How to visualise our social contacts as virtual avatars on wearable AR devices}, we explored two dimensions on the social AR continuum; 1) visualising social contacts (section \ref{sec:contacts:visualising}) and placement of social contacts (section \ref{sec:contacts:placing}). 

\begin{itemize}
    \item{We built HoloLens prototype visualising different level of details of social contacts based on proximity. The prototype allows user to change the social relationship with others by selecting and moving them closer or further away.}
    \item{We found statistical significant results in comparing visual fidelity with only proximity and no filter in terms of natural interaction, ease of use, and the ability to distinguish between social contacts.}
    \item{The preferred option was to combine both visual fidelity and proximity as filter for representing social contacts}
    \item{For placing social contacts there was no significant difference between viewing them as life-size avatars versus miniature representation. However both options scored above average in subjective questionnaire about usefulness, natural interaction and ease of use.}
\end{itemize}

\pagebreak
\section{Future Directions}
