%Mark reviewed on May 9th 2019
\chapter{Conclusions}
\label{ch:conclusions}

% Tobias: The chapter is mainly summarising and repeating the thesis which it should do. However, it is very much bullet point style, and I think you use it a bit too much (bullet points) and it makes it hard to read. Maybe you can briefly summarise in a text paragraph and only have the research highlights or the main lessons learned at the end of each paragraph.

% Tobias: I do not find enough on the general impact of your research and the future directions. I see that you discuss some directions, but I think there is way more. Maybe you can even present a sketched interface combining all of your findings to make it more visual. You could also use this to discuss the design space opened by your research. In general, I would still like to see more discussions, interpretations and thought-provoking ideas in this part. At the moment it is too much summarising what was said before (and only that). That leaves a bit of a "boring taste" while I think you would be better by using this last chapter for inspiration.

% Tobias: Maybe you even split the future work discussion into discussing limitations of the current work and how they could be overcome (this is the majority of what you write at the moment) and then you make a second part which is a bit more inspirational (maybe including this sketch I mentioned before if you are able to do that) where you outline your vision of integrating social networks and AR, where you currently think you are, and taking newest development and creative thinking, where technology could be in 5-10 years. If you do this, it is always good to show how the findings from this thesis still apply or how your research is relevant despite progressing technology. 


This thesis introduced the concept of the Social AR Continuum in Chapter \ref{ch:continuum} to answer important research questions about representing and interacting with social networks through wearable AR devices. This work explored the main dimensions of representing self and others in Chapter \ref{ch:contacts}, sharing data in Chapter \ref{ch:data} and annotation and interactions in Chapter \ref{ch:annotation}. 

The research focused on exploring how social proximity can be used to filter (i.e., show more/fewer details of) the representation of social contacts, shared virtual objects and the surrounding environments through wearable AR devices. In particular, this thesis addressed four research questions about 1) the dimension and variables of the Social AR Continuum, 2) the representation of self and others as virtual avatars on the social continuum, 3) the representation of data and shared environments, and 4) the interaction and annotation techniques used between social contacts.   

This work included building software prototypes to explore and validate different dimensions on the Social AR Continuum. User studies were conducted to evaluate the user interactions with these prototypes, the data collected were statistically analysed, and quantitative and qualitative results presented.

This chapter discusses the lessons learned from the user studies and summarises the general directions for future uses and developments of the Social AR Continuum. 

\pagebreak

\section{Lesson Learned}
% ---------------------------------------------------
\noindent
To answer \textit{\ref{rq:continuum}: What are the dimensions/factors/parameters in sharing social experiences on wearable AR devices?}, the concept and dimensions of the Social AR Continuum (Chapter \ref{ch:continuum}) was established.

\begin{itemize}
    \item{The dimensions of the Social AR Continuum can be used to vary shared social experiences based on social proximity. These dimensions include 1) self and others, 2) shared data and the surrounding environments, and 3) interaction and annotation.}
    \item{Future scenarios have been described by using the Social AR Continuum, which could help enhance the shared social AR experience. These include: 1) collaborative home decoration, 2) sharing a home office, 3) face-2-face social drinks, and 4) social connections at a conference.}
\end{itemize}

% ---------------------------------------------------
\noindent
To answer \textit{\ref{rq:people}: What dimensions work best for visualising and interacting with social contacts through wearable AR displays?}, this thesis explored two dimensions on the Social AR Continuum; 1) visualising social contacts (Section \ref{sec:contacts:visualising}) and placement of social contacts (Section \ref{sec:contacts:placing}). 
%  Mark: You may want to mention the focus group process here as well, and how people overwhelmingly wanted to use proximity for relationships.

\begin{itemize}
    \item{A Microsoft HoloLens prototype was built for visualising different levels of detail for social contacts based on proximity. The prototype allows users to change their social relationship with others by selecting and moving them closer or further away.}
    \item{A user focus group showed that people overwhelmingly wanted to use proximity for the social relationship when representing social contacts on wearable AR.}
    \item{Statistically significant results were found in comparing visual fidelity with only proximity and no filter in terms of natural interaction, ease of use, and the ability to distinguish between social contacts.}
    \item{The preferred option was to combine both visual fidelity and proximity as filters for representing social contacts}
    \item{For placing social contacts, there was no significant difference between viewing them as life-sized avatars versus miniatures. However, both options scored above average on a subjective questionnaire in terms of usefulness, natural interaction and ease of use.}
\end{itemize}

% ---------------------------------------------------
\noindent
To answer \textit{\ref{rq:data}: How does social proximity affect visualising and interaction with shared social data on wearable AR displays?}, this thesis explored filtering shared social 360-degree videos (Section \ref{sec:surrounding:360}), sharing 3D captured room surroundings (Section \ref{sec:surrounding:environment}), and the privacy concerns of hiding and showing parts of the shared room (Section \ref{sec:surrounding:hiding}). 

\begin{itemize}
    \item{System prototypes were developed on the Microsoft HoloLens for sharing and filtering social data such as 360-degree panorama video and 3D scanned room surroundings. The filtering was based on social proximity between social contacts.}
    \item{Results showed statistical significance confirming that filtering the shared social data was more preferred in terms of Social Presence.}
    \item{Perceived comfort in terms of privacy was statistically significant when a proximity filter was applied compared to no filter.}
    \item{There was a statistically significant difference in co-presence and comfort between using a proximity filter and no filter}
    \item{The privacy of shared spaces was more important for the sharer than for the viewer.}
    \item{The hiding mechanism had no significant difference in measuring ranking (user preference) between 1) removing, 2) blurring and 3) overlaying.}
\end{itemize}

% ---------------------------------------------------
\noindent
To answer \textit{\ref{rq:interaction}: How wearable AR displays can be used best for interacting with social contacts and shared social data?}, This thesis explored three concepts in sharing annotation and interaction for sharing social experiences. A user study was conducted on sharing social videos (Section \ref{sec:video}) and compared three conditions: List, AR, List+AR. This thesis looked into 1) creating and finding 3D annotations for social AR (Section \ref{sec:3D}) and into 2) sharing 360-degree panoramas for sharing social experiences (Section \ref{sec:pano}). 

\begin{itemize}
    \item{There was no statistically significant difference in usability between the List, AR and List+AR conditions for visualising comments on shared social videos. This indicates that all three conditions showed equally good usability for the user.}
    \item{There was a statistically significant difference in Social Presence of perceived message understanding and affective understanding between the List and AR conditions. This indicates that there was higher perceived message understanding and affective understanding in the AR and List+AR conditions compared to the List condition.}
    \item{There was a statistically significant difference in terms of ranking between the List and AR conditions, indicating that users preferred AR conditions over non-AR conditions.}
    \item{A prototype was developed to extend a normal 2D wearable (Google Glass) interface with a 3D sensor (Google Tango) to enable placing AR social annotations in real 3D spaces.}
    \item{Results found statistically significant differences indicating that using our 3D tagging system was easy and natural to use and was not found to be physically or mentally challenging. It was also found a previously created AR annotation was statistically significant in being useful.}
    \item{A prototype was developed connecting a wearable headset with a hand-held interface through a networking protocol to share a 360-degree panoramic image for sharing social experiences.}
    \item{A prototype for sharing social 360-degree panoramic images was developed on Google Glass and Android tablet users. The prototype was comparing three interaction techniques (audio, pointing and drawing) in terms of social presence, usability and ranking.}
    \item{Results found that the social presence was statistical significant difference in audio condition comparing to the drawing one for Google Glass users, but there was no difference for tablet users.}
\end{itemize}

%If you wanted to you could provide some overall design guideline recommendations for people building similar interfaces

%It might be good to add some paragraphs about research limitations and then use this to provide some input into possible future directions for research.

\section{Future Research Directions}

This work presented the concept of the Social AR Continuum and implemented several prototypes to explore the main dimensions on the continuum. User studies tested these dimensions. However, there are still many directions for future research. The following summarises some key opportunities for research in social AR.

The user studies in this thesis can be extended to include a larger number of participants. Large numbers of participants are important for social-related research, as most of our social interaction is driven and motivated by our social circles which have been established over years of connection and trust.  In addition, it would be good to run user studies in real social scenarios with actual groups of friends.

This research used a limited number of participants to reduce the complexity of running the studies, and in some cases, participants were not exactly representative of the subjects' own circles of friends. User studies in this thesis were strictly controlled where the scenario or situation was dictated to the participants with a predetermined task that simulated a social situation. Finally, in some cases, the remote participants were simulated, rather than having real people, which would be expected to have very different behaviours than actual users.

% Mark: I think it would be good to add a paragraph about how you could capture other data in the user studies. For example, do a conversational analysis of recorded conversation, or even measure physiological data.
This research was limited to measuring the social presence, and other subjective standard questionnaires defined by previous literature. Future studies could be extended to capture more data by including qualitative measures (e.g., analysis of recorded conversations) or recording bio-physiological data (e.g., heart rate measures), and how would those be affected by sharing social AR experiences.
 
In the future, it would be good to explore other dimensions of the Social AR Continuum that were not covered in this thesis, nor discovered due to advances in technology or the way people interact with each other. Ideas for new dimensions can be inspired by new interaction methods or based on imagination or works of fiction. It is important, however, to identify and validate any new potential continuum dimension with a validation method. A guideline process could be established for implementing a new dimension on the continuum and help with designing and developing applications for exploring new dimensions. 

There is also an opportunity to explore the interactions between this Social AR Continuum and other design paradigms, including other devices or platforms (e.g., hand-held, projected) and for different objectives other than social sharing (e.g., collaboration, business, play).

Finally, an interesting future direction would be to explore how the Social AR Continuum would be affected in extreme environments (e.g., Antarctica, outer space) with the limitations of connection bandwidth/speed and limited device availability. The exploration would need to include adaptability to various limitations of environments and interactions. 

Future directions also include avatar representation on a spectrum between humanoid or any shape representation for privacy or fun purposes. Another future direction is exploring the audio representation of social contact, especially when they are further away in terms of social proximity. Also, it will be interesting to explore other dimensions that were not covered here in this thesis — for instance, exploring different/same time and location for social sharing in AR, representing ageing in old relationships to remind us to stay in touch. 
